\begin{abstract}

Early noise introduces noise into the decision-making process at the time of stimulus encoding, as captured by encoding-decoding models of perception \cite{wei_efficient_2012}.
Recent work in visual perception has shown that priors can originate from our interactions with the natural environment \cite{girshick_cardinal_2011}.
The concept of a naturalistic prior has also been used to explain biases in numerosity judgements, where the prior follows a power law \cite{cheyette_unified_2020}.
This conflicts with Bayesian models of numerosity perception where the prior is allowed to adapt to the context \cite{prat-carrabin_efficient_2022}.
Here, we test the competing models of numerosity judgements by causally manipulating early noise and imposing biased priors.
We find a relationship between signal precision and response predictability which is only predicted by Bayesian models with adaptive priors.
Therefore, while naturalistic priors are plausible, they may only explain a subset of situations in which numerosity judgements are used, and models of numerosity judgement should allow priors to adapt to specific contexts.
    
\end{abstract}