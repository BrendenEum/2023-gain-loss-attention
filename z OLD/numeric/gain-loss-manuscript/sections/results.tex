\section*{Results} \label{section:results}

% ===============================================================================
\subsection*{Basic psychometrics} \label{section:basicpsychometrics}
% ===============================================================================

Accuracy, split by trials when blue or red were correct, can be seen in Fig. \ref{fig:accByColor}.
Although it may seem like there is a significant difference between the two types of trials, a t-test carried out at each stimulus duration shows that there is no significant difference between the two (minimum and maximum p-value = [\inputt{text-files/accuracyByColor_min_pval.txt}, \inputt{text-files/accuracyByColor_max_pval.txt}]). 

\begin{figure}[hbt!] 
    \centering
    \includegraphics[width=.75\textwidth]{figures/p.AccuracyByCorrectColor.pdf}
    \caption{Accuracy by correct color.
    Legend.
    } \label{fig:accByColor}
\end{figure}